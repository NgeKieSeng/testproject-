\documentclass[psamsfonts,a4paper]{amsart}
\usepackage{geometry}
 \geometry{
 a4paper,
 top=5mm,
 bottom=10mm
 }
\usepackage{amsmath, amsthm,  amsfonts, etoolbox, changepage,lipsum, bigints, relsize, mathtools, color, tikz-cd, amssymb}
\usepackage{aliascnt}
\usepackage{hyperref}
\usepackage{amsmath, amsthm,  amsfonts, etoolbox, changepage,lipsum, bigints, relsize, mathtools, color, tikz-cd, amssymb,   tikz, multirow ,adjustbox, tabularx, rotating, booktabs, lscape,xcolor,colortbl}
\usepackage{aliascnt}
\usepackage{hyperref}
\usepackage{cleveref}
\usepackage{url}
\usepackage[english]{babel}
\usepackage[utf8]{inputenc}
\input xypic
\xyoption{all}
\usepackage[all,arc]{xy}
\usepackage{enumerate}
\usepackage{mathrsfs}
\usepackage{chngcntr}
\counterwithin{table}{section}


\hypersetup{
    colorlinks=true,
    linkcolor=blue,
    filecolor=magenta,      
    urlcolor=cyan,
    pdftitle={These}
   } 
 
\urlstyle{same}

%\hoffset -25truemm
%\oddsidemargin=25truemm
   %\evensidemargin=25truemm
%\textwidth=157truemm
%\voffset -25truemm
%\topmargin=30truemm
%\headheight=0truemm
%\headsep=0truemm
%\textheight=235truemm



\hypersetup{
    colorlinks=true,
    linkcolor=blue,
    filecolor=magenta,      
    urlcolor=cyan,
    pdftitle={These}
   } 
 



  
  \theoremstyle{plain}
  \newtheorem{theorem}{Theorem}[section]
 \newtheorem{lemma}[theorem]{Lemma}
  \newtheorem{Proposition}[theorem]{Proposition}
  \newtheorem{question}[theorem]{Question}
 \newtheorem{example}[theorem]{Example}  
 \newtheorem{corollary}[theorem]{Corollary}
 \newtheorem{motivation}[theorem]{Motivation}
 \newtheorem{conjecture}[theorem]{Conjecture}
  
% \newtheorem{corollary}{Corollary}[section]
\newtheorem{proposition}[theorem]{Proposition}
%\newtheorem{lem}[theorem]{Lemma}
\crefname{lemma}{Lemma}{Lemma}
  \crefname{corollary}{Corollary}{Corollary}
  \crefname{theorem}{Theorem}{Theorem}
  \crefname{definition}{Definition}{Definition}
   \crefname{proposition}{Proposition}{Proposition}
 \crefname{section}{Section}{Section} 
   \crefname{construction}{Construction}{Construction}
   \crefname{generalization}{Generalization}{Generalization}
  \crefname{construction}{Construction}{Construction}
  \crefname{notation}{Notation}{Notation}
   \crefname{example}{Example}{Example}
  \crefname{remark}{Remark}{Remark}
  \crefname{fact}{Fact}{Fact}
  \crefname{conjecture}{Conjecture}{Conjecture}
  \crefname{motivation}{Motivation}{Motivation}  
  
  \newtheorem{definition}[theorem]{Definition}
  
  \newtheorem{remark}[theorem]{Remark}
%  \newtheorem{example}{Example}[section]
  \newtheorem{note}{Note}[section]
  \newtheorem{exercise}{Exercise}[section]
\newtheorem{generalization}[theorem]{Generalization}
\newtheorem{construction}[theorem]{Construction}
\newtheorem{notation}[theorem]{Notation}
\newtheorem{fact}[theorem]{Fact}

  \numberwithin{equation}{section}
  \numberwithin{figure}{section}



  % ------------------------------------------------------------------------

  % caligraphic
  \renewcommand{\cH}{{\mathcal H}}
  \newcommand{\cQ}{{\mathcal Q}}
  \newcommand{\cA}{{\mathcal A}}
  \newcommand{\cE}{{\mathcal E}}
  \newcommand{\cM}{{\mathcal M}}
  \renewcommand{\cL}{{\mathcal L}}
  \renewcommand{\cD}{{\mathcal D}}
  \newcommand{\cP}{{\mathcal P}}
  \newcommand{\cC}{{\mathcal C}}
  \newcommand{\cG}{{\mathcal G }}
  \newcommand{\cO}{{\mathcal O }}
  \newcommand{\cT}{{\mathcal T }}
  \newcommand{\cK}{{\mathcal K }}
  \newcommand{\cS}{{\mathcal S }}
  \newcommand{\cW}{{\mathcal W }}
  \newcommand{\GG}{{\mathfrak G}}
  \newcommand{\cB}{{\mathcal B }}


   \newcommand{\sta}{\stackrel}
   \newcommand{\ba}{\begin{eqnarray}}
   \newcommand{\na}{\end{eqnarray}}
   \newcommand{\ban}{\begin{eqnarray*}}
   \newcommand{\nan}{\end{eqnarray*}}

  %fraktur
  \newcommand{\g}{{\mathfrak g}}
 \newcommand{\s}{{\mathfrak s}}
 \newcommand{\fl}{{\mathfrak l}}
 \newcommand{\h}{{\mathfrak h}}
 \newcommand{\fW}{{\mathfrak W}}

  % math blackboard
  \newcommand{\C}{{\mathbb C}}
  \newcommand{\R}{{\mathbb R}}
  \newcommand{\Z}{{\mathbb Z}}
   \newcommand{\Q}{{\mathbb Q}}
   \newcommand{\N}{\mathbb N}
    \newcommand{\T}{\mathbb T}
       \newcommand{\bS}{\mathbb S}
        \newcommand{\W}{\mathbb W}
         \newcommand{\D}{\mathbb D}
 % \renewcommand{\A}{{\mathbb A}}

  % greek
  \renewcommand{\a}{\alpha}
  \renewcommand{\b}{\beta}
  \renewcommand{\c}{\gamma}
  \newcommand{\eps}{\epsilon}
  \renewcommand{\d}{\delta}
  
  %script
  \newcommand{\sA}{\mathscr{A}}
  \newcommand{\sS}{\mathscr{S}}
  \newcommand{\sB}{\mathscr{B}}
  \newcommand{\sC}{\mathscr{C}}
\newcommand{\sD}{\mathscr{D}}  
\newcommand{\sP}{\mathscr{P}}
  \newcommand{\sW}{\mathscr{W}}
  
  % miscellaneous
\newcommand{\ra}{\rightarrow}  
\newcommand{\da}{\downarrow}
    \newcommand{\disp}{\displaystyle}
\newcommand{\xra}{\xrightarrow}
\newcommand{\rat}{\rightarrowtail} 
\newcommand{\lat}{\leftarrowtail}
\newcommand{\rra}{\rightrightarrows}
\newcommand{\lla}{\leftleftarrows}
\newcommand{\hra}{\hookrightarrow}
\newcommand{\Lra}{\Leftrightarrow}
\newcommand{\Llra}{\Longleftrightarrow}
\newcommand{\tb}{\textbf} 
\newcommand{\bp}{\bigoplus} 
\newcommand{\bt}{\bigotimes} 
\newcommand{\seq}{\subseteq}
\newcommand{\mt}{\mapsto}
\newcommand{\Lgla}{\Longleftarrow}
\newcommand{\Lgra}{\Longrightarrow}
\newcommand{\sg}{\sigma}
\newcommand{\ot}{\otimes}
\newcommand{\op}{\oplus}

\newcommand{\overbar}[1]{\mkern 1.5mu\overline{\mkern-1.5mu#1\mkern-1.5mu}\mkern 1.5mu}



 
\def\cancel#1#2{\ooalign{$\hfil#1\mkern1mu/\hfil$\crcr$#1#2$}} 
\def\Dirac{\mathpalette\cancel D} 
\def\dirac{\mathpalette\cancel\partial} 
 \def\ub{\underline{\beta}}
  \def\ux{\underline{x}}
   \def\uy{\underline{y}}
    \def\uu{\underline{u}}
     \def\uv{\underline{v}}
       \def\uw{\underline{w}}
         \def\uz{\underline{z}}
         \def\ul{\underline{l}}
\def\e{\epsilon}
\def\check{\textbf{\textcolor{red}{CHECK! }}}
 \def\hB{\widehat{\cB_n}}



  \newcommand{\norm}[1]{\lVert {#1} \rVert}
  \newcommand{\RP}{\RR {\rm P}}
  \newcommand{\mar}[1]{{\marginpar{\textsf{#1}}}}
  \newcommand{\cstar}{\C^*}
\newcommand{\br}[1]{\textbf{\textcolor{red}{#1}}}



  \newcommand{\<}{\langle}
  \renewcommand{\>}{\rangle}


  \renewcommand{\note}[1]{
  \medskip
  \noindent{\bf Author note:} #1
  \medskip
  }



\usepackage{scalerel,stackengine}
\stackMath
\newcommand\reallywidehat[1]{%
\savestack{\tmpbox}{\stretchto{%
  \scaleto{%
    \scalerel*[\widthof{\ensuremath{#1}}]{\kern-.6pt\bigwedge\kern-.6pt}%
    {\rule[-\textheight/2]{1ex}{\textheight}}%WIDTH-LIMITED BIG WEDGE
  }{\textheight}% 
}{0.5ex}}%
\stackon[1pt]{#1}{\tmpbox}%
}
\parskip 1ex



\newtheorem*{theorem*}{Theorem}


\newtheoremstyle{named}{}{}{\itshape}{}{\bfseries}{.}{.5em}{\thmnote{#3's }#1}
\theoremstyle{named}
\newtheorem*{namedtheorem}{Theorem}

\newtheoremstyle{name}{}{}{\itshape}{}{\bfseries}{.}{.5em}{\thmnote{#3}#1}
\theoremstyle{name}
\newtheorem*{namedformula}{}



%\newaliascnt{lemma}{theorem}
%\aliascntresetthe{lemma}
%\providecommand*{\lemmaautorefname}{Lemma}


\renewcommand\qedsymbol{QED}

\makeatletter
\newcommand\xrightleftarrows[2][]{\ext@arrow 0099{\longrightleftarrowsfill@}{#1}{#2}}
\def\longrightleftarrowsfill@{\arrowfill@\leftarrow\relbar\rightarrow}
\makeatother


\title{Invertibility of $R_i$}
\author{Kie Seng Nge}
\date{April 1, 2018}

\begin{document}
\maketitle

 Exercise 2.1. Write an explicit formula fo the differential $d$ below
\[
 R_i  \ot R'_i = (0 \ra U_i\{1\}) \xra{d} A_n \op U_i\{-1\} \xra{d} U_i\{-1\} \ra 0)
\]

and check that the complex decomposes into a direct sum

\[
(0 \ra U_i\{1\} \xra{\textbf{1}} U_i\{1\} \ra 0) \op (0 \ra A_n \ra 0)
\op (0 \ra U_i\{-1\} \xra{\textbf{1}} U_i\{-1\} \ra 0)
\]

The first and the last summands are null-homotopic, implying that 

\[
R_i  \ot R'_i \cong (0 \ra A_n \ra 0) = A_n.
\]


\color{blue}\begin{proof}
Note that $R_i = (0 \ra U_i\{1\} \xra{\b_i} A_n \ra 0)$ and \mbox{$R'_i = (0 \ra A_n \xra{\gamma_i} U_i\{-1\} \ra  0) $} where $U_i = P_i \ot_{\Z} {}_{i}P,$ $\b_i$ takes $ x \ot y \in P_i \ot_{\Z} {}_{i}P$ to $xy \in A_n,$ and $\gamma_i$ takes $1$ to $(i-1|i)\ot(i|i-1) + (i+1|i)\ot(i|i+1) + X_i\ot(i) + (i)\ot X_i.$
Subsequently, the double complex corresponding to $R_i  \ot R'_i$ has the form 


\begin{center}


\begin{tikzcd}
  & 0 &0 \\
0 \arrow[r] & U_i \ot U_i   \arrow[u]  \arrow{r}{\b_i \ot \textbf{1}} & A_n \ot U_i\{-1\} \arrow[u] \arrow[r] & 0   \\
0 \arrow[r] & U_i\{1\} \ot A_n   \arrow{u}{\textbf{1} \ot \gamma_i}  \arrow{r}{\b_i \ot \textbf{1}} & A_n \ot A_n \arrow{u}{\textbf{1} \ot \gamma_i} \arrow[r] & 0  \\
  & 0\arrow[u] & 0\arrow[u] \\
\end{tikzcd}

\end{center}

Therefore, we get the following total complex

\[
R_i  \ot R'_i = 0 \ra U_i\{1\} \ot A_n \ra A_n \ot A_n \op U_i \ot U_i  \ra A_n \ot U_i\{-1\} \ra 0
\]

\noindent Note that 

\begin{tikzcd}
0 \arrow[r] & U_i\{1\} \ot A_n \arrow{r}{\begin{bmatrix}
\b_i \ot \textbf{1} \\ \textbf{1} \ot \gamma_i
\end{bmatrix} } \arrow{d}{\cong} & A_n \ot A_n \op U_i \ot U_i \arrow{r}{\begin{bmatrix}
-\textbf{1} \ot \gamma_i \hspace{1mm} \b_i \ot \textbf{1} 
\end{bmatrix} } \arrow{d}{\cong} & A_n \ot U_i\{-1\} \arrow[r] \arrow{d}{\cong} & 0 \\
0 \arrow[r] & U_i\{1\} \arrow{r}{d_1}  & A_n  \op U_i\{1\} \op U_i\{-1\} \arrow{r}{d_2}  &  U_i\{-1\} \arrow[r]  & 0 
\end{tikzcd}

The first vertical isomorphism from the left is followed from the fact that $U_i\{1\}$ is an $A_n-$bimodule. 
	Similar arguments applied to the third vertical isomorphism. 
	The second vertical isomorphism is obtained from Claim 1.7 by using the definition of $U_i$ and $A_n$ are tensored over itself.\\
	
	To see that those isomorphisms are actually chain map, we need to define $d_1$ and $d_2$ accordingly. 
	First, $d_1$ is a composition of maps 
	
\begin{align*}
x \ot y \in U_i\{1\} 
&\mapsto x \ot y \ot 1  \\
&\mapsto \b_i(x \ot y) \ot 1 + x \ot y \ot \gamma_i(1)  \\
& \hspace{2mm} = xy \ot 1 \\
&  + x \ot (i) \ot ((i-1|i)\ot (i|i-1) + (i+1|i)\ot(i|i+1) + X_i\ot(i) + (i)\ot X_i)y \\
& \mapsto xy + x \ot X_i \ot y + x \ot (i) \ot X_iy \\
& \mapsto xy + x \ot y + x \ot X_iy \in  A_n  \op U_i\{1\} \op U_i\{-1\},
\end{align*}

\noindent in other words, $d_1 = \begin{bmatrix}
\b_i \\ \textbf{1} \\ x \ot y \mapsto x \ot X_i y
\end{bmatrix}.$

Next, $d_2$ is a composition of maps 
\begin{align*}
a + x_1 \otimes_{\Z} y_1 + x_2 \otimes_{\Z} y_2 
& \mapsto 1 \otimes_{A_n} a + x_1 \otimes_{\Z} X_i \otimes_{A_n} y_1 + x_2 \otimes_{\Z} (i) \otimes_{A_n} y_2 \\
& \mapsto 1 \ot_{A_n} a + x_1 \ot (i| i-1) \ot (i-1|i) \ot y_1 + x_2 \ot (i) \ot (i) \ot y_2 \\
&\mapsto -  a \ot \gamma_i(1) + x_1X_i \ot (i) \ot y_1 + x_2 \ot (i) \ot y_2 \\
& \mapsto -a \gamma_i(1) + x_1X_i \ot y_1 + x_2 \ot y_2,
\end{align*}

\noindent in other words, $d_2$ = $\begin{bmatrix}
-\gamma_i & x \ot y \mapsto xX_i \ot y & \tb{1}
\end{bmatrix}$.
	They are chain map because
	
\begin{align*}
d_2 d_1(x \ot y)
 &= \begin{bmatrix}
-\gamma_i & x \ot y \mapsto xX_i \ot y & \tb{1}
\end{bmatrix} 
\begin{bmatrix}
\b_i \\ \textbf{1} \\ x \ot y \mapsto x \ot X_i y
\end{bmatrix}
(x \ot y)  \\
&= \begin{bmatrix}
-\gamma_i & x \ot y \mapsto xX_i \ot y & \tb{1}\end{bmatrix} 
\begin{bmatrix}
xy \\   x \ot y \\  x \ot X_iy
\end{bmatrix} \\
&= -x\gamma_i(1)y + xX_i \ot y + x \ot X_iy \\
&= -x((i-1|i)\ot(i|i-1) + (i+1|i)\ot(i|i+1) + X_i\ot(i) + (i)\ot X_i)y \\
& \hspace{2mm}  + xX_i \ot y + x \ot X_iy\\
& = 0 \hspace{10mm} \text{by the definition of $x$ and $y$}
\end{align*}

After that, we need to check the existence of isomorphism between the chain complexes so that the complex decomposes into a direct sum as described before.

\begin{center}


\begin{tikzcd}
0 \arrow[r] & U_i\{1\}\arrow{r}{\begin{bmatrix}
\b_i \\ \textbf{1} \\ x \ot y \mapsto x \ot X_i y
\end{bmatrix}} \arrow{d}{\cong} & A_n  \op U_i\{1\} \op U_i\{-1\} \arrow{r}{\begin{bmatrix}
-\gamma_i \hspace{1mm} x \ot y \mapsto xX_i \ot y \hspace{1mm} \tb{1}
\end{bmatrix}} \arrow{d}{\cong} &  U_i\{-1\} \arrow[r] \arrow{d}{\cong} & 0 \\
0 \arrow[r] & U_i\{1\} \arrow{r}{\begin{bmatrix}
0 \\ \textbf{1} \\ 0
\end{bmatrix}}  & A_n  \op U_i\{1\} \op U_i\{-1\} \arrow{r}{\begin{bmatrix}
0 \hspace{1mm} 0 \hspace{1mm} \tb{1}
\end{bmatrix}}   &  U_i\{-1\} \arrow[r]  & 0 
\end{tikzcd}
\end{center}


It can be verified that we can take the first and third map as the identity map \tb{1} and the second map as $\begin{bmatrix}
\textbf{1} & -\b_i & 0 \\
0 & \textbf{1} & 0 \\
-\gamma_i & x \ot y \mapsto xX_i \ot y & \textbf{1}
\end{bmatrix}$, so that the two chain complexes are isomorphic to each other.\\

Finally, since the second map is a direct sum of three chain complexes, namely $(0 \ra U_i\{1\} \xra{\tb{1}} U_i\{1\} \ra 0),$ $(0 \ra A_n \ra 0),$ and $(0 \ra U_i\{-1\} \xra{\tb{1}} U_i\{-1\} \ra 0).$ 
	But, the first and the last chain complexes are null-homotopic. 
	Consequently, $R_i \ot R'_i \cong A_n.$

\end{proof}


\color{black}
\begin{lemma}[Gaussian elimination]
Let $X, Y, Z, W, U, V$ be six objects in  an additive category and consider a complex

$$\cdots \ra U \xra{u} X \op Y \xra{f} Z \op W \xra{v} V \ra \cdots $$

where $f = \begin{pmatrix}
A && B \\
C && D
\end{pmatrix}$ and $u,v$ are arbitrary morphisms.
	If $D: Y \ra W$ is an isomorphism, then the complex above is homotopic to a complex
	$$\cdots \ra U \xra{u} X  \xra{A-BD^{-1}C} Z \xra{v|_Z} V \ra \cdots $$

\end{lemma}

\color{blue}
\begin{proof}
The key is the following commutative diagram:

\begin{center}
\begin{tikzcd}
\cdots \arrow{r} & U \arrow{r}{u} \arrow{d}{id}  & X \op Y \arrow{r}{f} \arrow{d}{\a} & Z \op W \arrow{r}{v} \arrow{d}{\b}  & V \arrow{r} \arrow{d}{id} & \cdots \\
\cdots \arrow{r} & U \arrow{r}{\a u}  & X \op Y \arrow{r}{g} & Z \op W \arrow{r}{v \b^{-1}}  & V \arrow{r} & \cdots
\end{tikzcd}
\end{center}

where 

$$g = \begin{pmatrix}
A-BD^{-1}C && 0 \\
0  && D
\end{pmatrix}  \hspace{5mm}
\a = \begin{pmatrix}
1  &&  0 \\
D^{-1}C  && 1
\end{pmatrix}  \hspace{5mm}
\b = \begin{pmatrix}
1 && -BD^{-1}  \\
0 && 1 
\end{pmatrix}.$$
The vertical map of complexes is a homotopy equivalence with 

$$
\a^{-1} = \begin{pmatrix}
1  &&  0 \\
-D^{-1}C  && 1
\end{pmatrix}  \hspace{5mm} \text{and} \hspace{5mm} 
\b^{-1} = \begin{pmatrix}
1 && BD^{-1}  \\
0 && 1 
\end{pmatrix}.$$


 It is straightforward to check that the bottom row is homotopic to 
$$\cdots \ra U \xra{u} X  \xra{A-BD^{-1}C} Z \xra{v|_Z} V \ra \cdots ,$$
namely, 


\begin{center}
\begin{tikzcd}
\cdots \arrow{r} & U \arrow{r}{\a u} \arrow{d}{id} & X \op Y \arrow{r}{g} \arrow{d}{\begin{bmatrix}
1 \hspace{2mm} 0 \end{bmatrix}}  \arrow[ldd,dotted,"0"] & Z \op W \arrow{r}{v \b^{-1}} \arrow{d}{\begin{bmatrix}
1 \hspace{2mm} 0 \end{bmatrix}}  \arrow[ldd,dotted,"s"] & V \arrow{r} \arrow{d}{id} \arrow[ldd,dotted,"0"] & \cdots \\
\cdots \arrow{r} &  U \arrow{r}{u}  \arrow{d}{id} & X  \arrow{r}{A-BD^{-1}C}  \arrow{d}{\begin{bmatrix}
1 \\ 0 \end{bmatrix}} \arrow[ldd,dotted,"0"] & Z \arrow{r}{v|_Z} \arrow{d}[swap]{\begin{bmatrix}
1 \\ 0 \end{bmatrix}} \arrow[ldd,dotted,"0"] & V \arrow{r} \arrow{d}{id} \arrow[ldd,dotted,"0"] &\cdots\\
\cdots \arrow{r} & U \arrow{r}{\a u} \arrow{d}{id} & X \op Y \arrow{r}{g} \arrow{d}[swap]{\begin{bmatrix}
1 \hspace{2mm} 0 \end{bmatrix}}   & Z \op W \arrow{r}{v \b^{-1}} \arrow{d}[swap]{\begin{bmatrix}
1 \hspace{2mm} 0 \end{bmatrix}}   & V \arrow{r} \arrow{d}{id} & \cdots\\
\cdots \arrow{r} &  U \arrow{r}{u}  & X  \arrow{r}{A-BD^{-1}C}  & Z \arrow{r}{v|_Z} & V \arrow{r}  &\cdots
\end{tikzcd}
\end{center}
where $s = \begin{bmatrix}
0 & 0 \\ 
0 & -D^{-1} \end{bmatrix}.$
To see this, note that 

$$\begin{bmatrix}
0 & 0 \\ 0 & -1 \end{bmatrix} =\begin{bmatrix}
1 \\ 0 \end{bmatrix} \begin{bmatrix}
1 & 0 \end{bmatrix} - id
=  s \circ g + \a u \circ 0 
=  \begin{bmatrix}
0 & 0 \\ 
0 & -D^{-1} \end{bmatrix} 
\begin{bmatrix}
A-BD^{-1}C && 0 \\
0  && D
\end{bmatrix} $$
and

$$ 0 = \begin{bmatrix}
1 & 0 \end{bmatrix}  \begin{bmatrix}
1 \\ 0 \end{bmatrix} 
 - id = 0 \circ (A-BD^{-1}C) + u \circ 0,  $$
as desired. 

\end{proof}

\vspace{10mm}

\begin{proof}[Second proof of Exercise 2.1]
In the light of lemma 6.1, the complex\\

\begin{center}
\begin{tikzcd}
0\arrow[r] & U_i\{1\}\arrow{r}{\begin{bmatrix}
\b_i \\ \textbf{1} \\ x \ot y \mapsto x \ot X_i y
\end{bmatrix}}  & A_n  \op U_i\{1\} \op U_i\{-1\} \arrow{r}{\begin{bmatrix}
-\gamma_i \hspace{1mm} x \ot y \mapsto xX_i \ot y \hspace{1mm} \tb{1}
\end{bmatrix}} &  U_i\{-1\} \arrow[r]  & 0 
\end{tikzcd}
\end{center}
is homotopic to 

$$ 0 \ra A_n \ra 0.$$

\end{proof}

\section{Git}


\end{document}